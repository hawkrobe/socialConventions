
\documentclass[11pt, oneside]{article}   	% use "amsart" instead of "article" for AMSLaTeX format
\usepackage{geometry}                		% See geometry.pdf to learn the layout options. There are lots.
\geometry{letterpaper}                   		% ... or a4paper or a5paper or ... 
%\geometry{landscape}                		% Activate for rotated page geometry
%\usepackage[parfill]{parskip}    		% Activate to begin paragraphs with an empty line rather than an indent
\usepackage{graphicx}				% Use pdf, png, jpg, or eps§ with pdflatex; use eps in DVI mode
\usepackage{multirow}
\pagenumbering{gobble}% Remove page numbers (and reset to 1)

		\usepackage{amssymb,amsfonts,amsmath}
\usepackage{textcomp}
\usepackage{setspace}
\usepackage{multirow}
\usepackage{array}				% TeX will automatically convert eps --> pdf in pdflatex		
\usepackage{amssymb}

%SetFonts

%SetFonts

\begin{document}

\noindent\textbf{Computing stability}

\setstretch{2} 

Consider a single pair of players' time series of game outcomes, which can be encoded as a sequence of states $S$. Each $s \in S$ is drawn from the alphabet $\Sigma = \{R, B, T\}$, with $R$ indicating `red wins', $B$ indicating `blue wins' and $T$ indicating `tie.' This is an exhaustive set of outcomes. Let $S_m$ be the set of all subsequences of length $m$. We train a Markov Chain of order $m$ on this sequence of states such that $$\widehat{P}(X_t | X_{t-1}, \dots X_{t-m}) = \frac{\sum_{s\in S_m} \mathbb{I}_{\{s = X_{t-m}\cdots X_{t-1}X_t\}} + 1/3}{\sum_{s\in S_{m-1}} \mathbb{I}_{\{s = X_{t-m}\cdots X_{t-1} + 1\}}}$$ where $\mathbb{I}_{b}$ is the indicator function returning 1 when $b$ is true. Note that by adding `virtual counts' for each state in the alphabet, we are using the Bayesian maximum a priori (MAP) estimator with a uniform prior over each conditional distribution. This prevents us from overreliance on too little data. Note that as the sample size grows, the virtual counts will contribute less and less to the estimate, such that this estimator converges to the maximum likelihood estimator \cite{YoungSmith05_StatisticalInference}.

Once we learn these conditional probabilities, we can compute Shannon's (conditional) surprisal  $$S(t) = -\log_2[P(x_t | x_{t-1}, \dots, x_{t-m})]$$ for each step $t$ in the time series, where $x_t$ is the outcome and $P(x_t | x_{t-1}, \dots, x_{t-m})$ is the probability of that outcome, according to the Markov chain \cite{Shannon48}. This creates a second time series consisting of surprisals for each outcome given the entire time course of coupled dynamics. Note that a stable equilibrium shows up as a long period of low surprisal in this time series. 

Following this observation, we define the \emph{stability} for a particular pair of participants as the distribution of surprisals their outcomes generate (see S5 Fig. for examples of each step in this process). To compare conditions at the group level, we aggregated the total set of surprisals computed for a condition into one long list (e.g. for the high-discrepancy, dynamic condition, there are 69 pairs and $50 - m$ surprisal values for each pair, so for an order 2 Markov chain, the condition as a whole has 3312 surprisal values), and take the mean. The resulting stability for each condition was consistent across different choices of $m$, the `memory' of the process (see S6 Fig.) See S4 Fig. for a qualitative comparison of the surprisal CDFs across all conditions.

\bibliography{../plos}
\bibliographystyle{../plos2015}


\end{document}  